\section{Task 2 Performance Modelling of Shared Memory Programs}

\subsection{Modelling}
The overall cost of the program can be divided into sequential part cost and parallel part cost.
\[
	t = t_{seq} + t_{par}
\]
However, the computational load of sequential part is much smaller than the parallel part,
so we neglect this term safely. 
The parallel term is constituted by four parts, including parallel region entry/exist cost $t_s$,
the cost of double words of cache miss for coherent read $t_{w,R}$,
the cost of double words of cache miss for coherent write $t_{w,W}$
and the cost of computation $t_f$.

Consequently, the overall cost can be modelled as 
\begin{eqnarray*}
	t = n_E t_s + n_R t_{w,R} + n_W t_{w,W} + n_f t_f + \textrm{cache misses from halo part} \\
	 = 4\times r \cdot t_s + \frac{M\cdot Q}{8}\cdot (2 \times r - 1)  \cdot t_{w,R} + Q \cdot (2 \times r -1 ) \cdot t_{w,W} + r\cdot \frac{M\cdot N}{Q} \cdot t_f
 \end{eqnarray*}
where $n_E, n_R, n_W, n_f$ is the number of occurences of corresponding terms.
We neglect the cache misses from halo part, since this term is much smaller than those of non-halo part.
Since we both parallel the boundary update loops and advection field update loops,
$n_E$ is $4\cdot r$.
Since in the maximize performance configuration, we divide the whole field into blocks in column direction,
the read cache miss occurs on the boundary of two adjacent blocks,
and each cache line has $8$ doubles, so $n_R = \frac{M\cdot Q}{8}\cdot (2 \times r - 1)$,
the $( 2 \times r - 1)$ reflects the fact that in the first r loop iteration, there is no coherent
read cache miss of u field. The coherent write cache miss comes from two adjacent block shares 
one cache line, so there will be one write cache miss between two blocks, the term $(2\times r-1)$
is similar to that in $t_{w,R}$ term, so $n_W = Q \cdot (2 \times r -1 )$. Finally, the $t_f$ contains
$23$ FLOPS and serveral interger operations.

\subsection{Measurement}
 
Firstly, we measure the parallel region entry/exits time $t_s$.
We run two experiments with $M=N=20$ and $r=100$. In the first experiment, the OpenMP will not be used,
In this case, the time cost is $2.5\times 10^{-4}$s.
While in the second experiment, we will use OpenMP with $p=1$.
In this case, the time cost is $5.4\times 10^{-4}$s.
Consequently, the $t_s = (5.4-2.5)\times 10^{-4} / r = 2.9 \times 10^{-6}$s.

Next we will measure the $t_{w,R}$ and $t_{w,W}$. Since the $t_{w,W}$ term will not depend on $M$, but $t_{w,R}$ does,
so we perform two set of experiments with $p=8$ and $p=16$. 
For each set, we keep $N=2000, r=10000$, but varying $M = 1000, 2000, 3000$.

The experiment results are shown in Tab \ref{tab2_1}

\begin{table}[h]
	\centering
	\caption{Measure cost of coherent cache miss}
	\label{tab2_1}
	\begin{tabular}{lllll}
		\hline
		             & $M = 1000$ & $M = 2000$ & $M = 3000$ \\ \hline
		$p=8$            & $22.33$s &  $49.03$s &   $73.16$s \\
		$p=16$           & $8.764$s & $24.37$s & $39.12$s  \\ \hline
	\end{tabular}
\end{table}

From above results, we can get for inner socket ($p=8$), $t_{w,R} = 1.2\times 10^{-6}$s,
and for intra socket ($p=16$), $t_{w,R} = 3.2\times 10^{-7}$s. In fact, intra socket time should be larger than inner socket time. $t_{w,W}$ is negative in both cases. These un-reasonable results are due to my inappriciate choose of $M$ and $N$. 

I have no time to choose a set of better $M,N,r$.

\subsection{Experiments with $p=8$ and $p=16$}

\subsection{Prediction}
